\documentclass[aps,prd,notitlepage,floatfix,superscriptaddress,groupedaddress,nofootinbib]{revtex4-1}
\usepackage{graphicx}  % needed for figures
\usepackage{dcolumn}   % needed for some tables
\usepackage{bm}        % for math
\usepackage{amssymb}   % for math
\usepackage{amsmath}
\usepackage{subcaption}
\usepackage{xcolor}
\usepackage{soul}
\usepackage{natbib}
% \usepackage{booktabs}
\usepackage[singlelinecheck=false, justification=centering]{caption}
\usepackage{listings}
\usepackage{float}
\usepackage{enumitem}
\usepackage{tabularx}
\lstset{basicstyle=\ttfamily,
  showstringspaces=false,
  commentstyle=\color{red},
  keywordstyle=\color{blue}
}

\usepackage{url}
\def\UrlBreaks{\do\/\do-}
\usepackage{breakurl}
\usepackage[breaklinks]{hyperref}

\usepackage[autostyle]{csquotes}

\usepackage[printonlyused, nohyperlinks, withpage]{acronym}

\newcommand{\jscomment}[1]{\colorbox{gray}{\textcolor{yellow}{#1}}}
\newcommand{\jsedit}[2]{\jscomment{\st{#1}}\jscomment{#2}}


\begin{document}

\title{Report on Udacity AI Nanodegree project 3:\\Adversarial search}

\author{Hans-Peter Schadler}
% \affiliation{Austria}

\date{\today}


\begin{abstract}
    In this project we are developing an AI agent for the game Isolation, a fully observable, adversarial and deterministic board game.
    We will implement the MINIMAX algorithm with Alpha-Beta pruning and analyze the performance of various search techniques and heuristics for the agent.
\end{abstract}

% \pacs{}
\maketitle

% \tableofcontents

% \newpage

% \makeatletter
% \let\toc@pre\relax
% \let\toc@post\relax
% \makeatother

% \newpage
\section{Problem analysis}

The assignment is to implement a MINIMAX agent for the game Isolation in \textit{custom\_agent.py}
and then pick one of three experiments (1. Custom heuristics, 2. Opening book or 3. Advanced search techniques).
For this assignment I decided to pick the first topic, developing a custom heuristic and perform various analyses to
assess its effectiveness.

All the evaluation runs where performed on a Dell XPS 13 with the following specs:
\begin{itemize}[nosep]
    \item Linux Mint 20.2 Cinnamon
    \item Kernel version 5.14.0-1032-oem
    \item Intel Core i7-1165G7 @ 2.80GHz (4 cores + HT)
    \item 16 GiB memory
\end{itemize}
If not otherwise mentioned, the \textit{run\_search.py} script was always ran with the default
time limit of 150ms and with the \textit{fair\_matches.py} option enabled.

\section{Definition of the heuristics}

In this section, we will define and describe the various heuristics that we tested during this project.

\subsection{Baseline}
The baseline heuristics is defined by the number of player moves - number of opp moves. It is the heuristics implemented in the sample player MINIMAX algorithm and is implemented also for our custom player in a similar manner. This will serve as the baseline for our comparison.

\begin{lstlisting}[language=python]
def heuristics_liberties(state: Isolation, player: int):
    """# player_moves - # opp_moves"""
    own_loc = state.locs[player]
    opp_loc = state.locs[1 - player]
    own_liberties = state.liberties(own_loc)
    opp_liberties = state.liberties(opp_loc)
    return len(own_liberties) - len(opp_liberties)
\end{lstlisting}

\subsection*{Number of player moves only}
To define an even simpler heuristics, we use only the number of player moves possible from the current position.

\begin{lstlisting}[language=python]
def heuristics_liberties_player_only(state: Isolation, player: int):
    """Only # player moves"""
    own_loc = state.locs[player]
    own_liberties = state.liberties(own_loc)
    return len(own_liberties)
\end{lstlisting}

\subsection*{Number of opponent moves only}
A simple heuristic only considering number of opponent moves

\subsection*{Number of play counts: Favor longer games}
Prioritize higher play counts (longer games)

\subsection*{Number of play counts: Favor shorter games}
Prioritize lower play counts (shorter games)

\subsection*{Baseline + keep the opponent close}
Baseline while keeping the enemy close.

\subsection*{Baseline + keep the opponent far}
Baseline while keeping the enemy far.

\subsection*{``Deeper'' baseline heuristics}
Evaluate number of possible moves from current and next positions

\subsection*{Conservative baseline}
Own moves are more important than enemy moves

\subsection*{Aggressive baseline}
Prefer tactics where we starve the opponents moves

\section{Results}

\section{Influence of the search depth on the win percentage}

\section{Conclusion}

% \begin{figure}[tbh!]
%     %\vspace*{-0.5em}
%     \centering
%     \includegraphics[width=0.7\linewidth]{images/problem_1_time.pdf}
%     \caption{Runtime vs. algorithm for problem 1}
%     \label{fig:p1_runtime}
% \end{figure}

% \appendix

% \section{\label{app:some_appendix}Some appendix}

% \newpage
\bibliographystyle{unsrtnat}
%\bibliographystyle{plain}
\bibliography{references}

\end{document}